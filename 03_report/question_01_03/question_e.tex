
Question 1.3e:\\	
\textsl{Even if a test is shown as significant by replication of the same experiment, we still cannot make a scientific claim.}\\
	
\textsl{True or False? }\\

Answer:\\
Given the scenario that there was an significant experiment which should be confirmed in a second study several considerations should be made to report a scientific claim. Firstly the experimentation setup must be robustly executed. If a too small sample size or poor variables are measured, there is no improvement of the claim just by repeating the experiment. For example, if we state that a states caffeine consumption leads to higher income by measuring both, this could result in the same outcome. But here are questionable assumptions made, which are rooted in the experimentation setup. This scenario might actually lead to significance, but it should not be published.\\

Secondly, the general rule should be considered, that the statistical significance is not the only criterion on which scientific claims should be made. Even it is an important objective factor and over all domains accepted, there should always be an awareness, that a type I error is being made or more in general there is a statistical significance despite the conclusion was drawn incorrectly.\\

To cut a long story short: There is no simply "True or False" to this question. The answer depends on the circumstances and is not True in general. But testing significant experiments can be an important tool to confirm or refute hypotheses in a scientific way.\\

%old answer:
%The weakness in the conclusion of question 1.3d was, that there could no statistical relevance be proven. But with respect to a replication of the experiment, things might be judged differently. But again, a small sample size can reduce the impact of the study. If the first experiment was not significant, whereas the second was, our p-value is close to our significance level. But here we can hardly judge the magnitude.\\

%If the sample size is very large, e.g. by collecting data of multiple comparable companies, and the test is still significant, this underlines the theory which was concluded in the task of question 1.3d. If there are no formal restrictions due to missing significance, and the theory follows a hypothesis in a logical manner, there is no reason why this experiment should not be scientifically claimed.




