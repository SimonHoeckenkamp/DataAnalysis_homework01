
Question 1.3g:\\	
\textsl{If I see a significant p-values, it could be the case that the null hypothesis is consistent with truth, but my statistical model does not match reality.}\\
	
\textsl{True or False?}\\

Answer:\\
This question can be answered with True.\\

Of course the null-hypothesis can be True, even if the p-value indicates that it should be rejected. This is almost the definition of the p-value: Probability that the test outcome is at least as extreme as given in the study.\\

Another point is mentioned in the article "ASA Statement on Statistical Significance and P-Values". If the p-value indicates correctly, that the null-hypothesis should be rejected there is still a chance, that the hypothesis itself is not correct. For example if a states caffeine consumption correlates with it's income, there might be a correlation but one can hardly explain the causality. Therefore the reality is different of what the experiment would indicate.\\

