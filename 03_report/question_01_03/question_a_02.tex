
Question 1.3a-2:\\	
\textsl{Your friend hears your point, and think it makes sense. He also hears about that with more data, relations are less likely to be observed just by chance, and inference becomes more accurate. He asks, if he gets more and more data, will the procedure he proposes find the true effects?}\\

Answer:\\
Indeed, with an increasing number of data points, the reproducibility increases. But just like Question 1.3a-1 indicated, a low p-value must not be mistaken for the "truth".\\

Besides the high number of data points a good measurement quality and solid experimentation setup in general is fundamental. But with more data in terms of number of features the chance of coincidental significances increases as well.\\

The p-value, which shows how good the data fits to a "specified hypothetical explanation", might lead to important insights. But it should not be mistaken with a true relationship between a cause and it's consequences. If the significance can be supported in a reproducible manner and if it follows logical rules, it can be an very important indicator for an significant behavior of the data with respect to specific hypotheses.\\





