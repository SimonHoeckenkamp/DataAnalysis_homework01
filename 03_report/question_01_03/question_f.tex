
Question 1.3f:\\	
\textsl{Your lab mate is writing up his paper. He says if he reports all the tests and hypothesis he has done, the results will be too long, so he wants to report only the statistical significant ones.}\\
	
\textsl{Is this OK? If not, why?}\\

Answer:\\
The person who is following this approach should better explain what he did and why he did it in that way. Without the information about the whole experiment the results are judged in a different manner. As the article "ASA Statement on Statistical Significance and P-Values" states, "p-values and related analyses should not be reported selectively".\\

If he wants to publish his results he should follow the instructions of the above mentioned article: "Valid scientific conclusions based on p-values and related statistics cannot be drawn without at least knowing how many and which analyses were conducted, and how those analyses (including p-values) were selected for
reporting."\\

Only by following this rules and analyzing the chance of mentioning an significant event due to a high number of variables by giving the FWER (Family-Wise Error Rate) or the FDR (False Discovery Rate) he can publish his claims.


