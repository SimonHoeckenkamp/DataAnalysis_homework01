
Question 1.3d:\\	
\textsl{Your boss wants to decide on company's spending next year. He thinks letting each committee debates and propose the budget is too subjective a process and the company should learn from its past and let the fact talk. He gives you the data on expenditure in different sectors and the company's revenue for the past 25 years. You run a regression of the revenue on the spending on HR sector, and find a large effect, but the effect is not statistically significant. Your boss saw the result and says “Oh, then we shouldn't increase our spending on HR then".}\\
	
\textsl{Is his reasoning right?}\\

Answer:\\
Given that the analysis is based on annual balances, we have 25 data points on which we run our regression analysis. Therefore one cannot rely on trends from the statistical point of view. Here is where the knowledge of the specific area gets important.\\

If the data gives information about correlations in the data set, it is the person with domain knowledge, who has to draw the conclusion. In this example the data scientist may say that there is an correlation between variables, even if it is not statistically significant. But the person in charge has to decide whether this connection is meaningful or not.\\

One conclusion could be that the boss rejects investments in HR because of the possibility, that this correlation is due to randomness. But in this example he should be informed that it could still be a possibility to increase the revenue.\\




