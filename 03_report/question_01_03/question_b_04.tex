
Question 1.3b-4:\\	
\textsl{The lab runs a randomized experiment on 100 mice, add chocolate in half of the mice's diet and add in another food of the equivalent calories in another half's diet. They find that the difference between the two groups time in solving a maze puzzle has p-value lower then 0.05. Should they conclude that chocolate consumption leads to improved cognitive power in mice?}\\

Answer:\\
The general approach of this study seems reasonable. The test is set up, such that the mice's behavior after the consumption of chocolate can be analyzed. And in contrast to earlier questions the connections between the variables seems more comprehensible.\\

But the number of tested individuals is quite small. Therefore chances are high that a significant result occurred by coincidence. The p-value is an easy to use tool for checking if the null-hypothesis should be rejected. Drawing the conclusion only based on the p-value should be avoided, as mentioned in the article "ASA Statement on Statistical Significance and P-Values".\\

The insight won in this study could be a good starting point for further investigations, but should not be presented as the ultimate truth.\\
