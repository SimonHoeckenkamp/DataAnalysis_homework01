
Question 1.3b-3:\\	
\textsl{In order to study the relation between chocolate consumption and intelligence, what can they do?}\\

Answer:\\
To analyze the connection between chocolate consumption and intelligence, the architecture of the experiment should firstly focus on the data measurement, which is demonstrable valid for this purpose. The consumption of chocolate can either be defined in a (double-blind) test or figured out in a survey.\\

If the researchers want to show that acute chocolate consumption is beneficial for the intelligence, the double-blind test should be used. The participant gets in one time episode chocolate and in the next one an adequate replacement. After each episode a test should be done to measure the intelligence as objectively as possible (e.g. IQ-test).\\

If a connection between long-term chocolate consumption and the resulting intelligence of individuals should be analyzed, a survey with a corresponding intelligence test should be executed. But here an comparable control group is necessary. Further variables should be reported as well to show, that different individuals only can be separated with respect to the consumed chocolate.\\

Either way, the number of data points should be high enough to get robust and reproducible outcomes.\\

