
Question 1.3b-1:\\	
\textsl{A economist collects data on many nation-wise variables and surprisingly find that if they run a regression between chocolate consumption and number of Nobel prize laureates, the coefficient to be statistically significant. Should he conclude that there exists a relationship between Nobel prize and chocolate consumption?}\\

Answer:\\
If the economist would solely rely on the p-value to support his thesis, he would probably find enough "evidence" to make him believe that there is a relationship between chocolate consumption an the number of won Nobel prizes. But by doing this the economist would underestimate possible other explanations and might draw poor conclusions.\\

In this example one cause of high values in both features (chocolate consumption and number of won Nobel prizes) might be the income of a state. If it is high, more money can be invested in education and research. Equally there can be more money spent on luxury products like candies. This connection would not be drawn by the economist solely by expressing the significance in terms of a p-value. Therefore the correlation should not be mistaken with causality.\\

This is a good example of a study, in which the p-value is weighted to highly up to the degree, where it is the only explanation of a relationship. But just as in the article "ASA Statement on Statistical Significance and P-Values" mentioned, there are more factors to consider, when a result is to be presented.\\





