
Question 1.3a-1:\\	
\textsl{Your colleague on education studies really cares about what can improve the education outcome in early childhood. He thinks the ideal planning should be to include as much variables as possible and regress children's educational outcome on the set. Then we select the variables that are shown to be statistically significant and inform the policy makers. Is this approach likely to produce the intended good policies?}\\

Answer:\\
This approach has a high chance of producing results, which indicate dependencies between some variables and the early childhood education. But the essence of the article "ASA Statement on Statistical Significance and P-Values" states that the exclusive use of the statistical significance as parameter on which policies are evaluated is a poor approach.\\

Nevertheless, the sampling of large data sets with multiple different variables might lead to assumptions, which are supported by a p-value below the widely used limit of 0.05. But the scientific research should not stop with this discovery. There is still a good chance that the outcome happened because of coincidence. Therefore the low p-value must not be mistaken for "truth".\\

In the article mentioned above this analysis based on the p-value is only one part, on which decisions or scientific researches should be based on. A robust experimentation setup is the first step, which includes a sufficient high number of data points and good quality of measurements. After finding significances in the data, the results should be understood and analyzed in one scope. If the results follow a logical and reproducible manner, there is no reason, why policies should not consider new insights.\\
