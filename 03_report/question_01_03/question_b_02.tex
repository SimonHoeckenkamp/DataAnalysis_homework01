
Question 1.3b-2:\\	
\textsl{A neuroscience lab is interested in how consumption of sugar and coco may effect development of intelligence and brain growth. They collect data on chocolate consumption and number of Nobel prize laureates in each nation, and finds the correlation to be statistically significant. Should they conclude that there exists a relationship between chocolate consumption and intelligence?}\\

Answer:\\
The neuroscience lab wants to state that there is an relationship between the consumption of chocolate and intelligence by analyzing a states consumption and the number of Nobel price laureates.\\

But despite the correlation between these two variables this approach lacks of some important aspects. Firstly, the researchers use the number of won Nobel prices as a variable for intelligence. Even if mostly smart scientists are winning the award, there is no reason to assume, that even smarter people exist, who do not win a Nobel price. Besides intelligence, environmental factors like education and research funds lead to superior scientific results. Therefore the number of Nobel laureates should not be used as an equivalent for intelligence.\\

Even if we take the stated connection (number of Nobel prices and intelligence) as given, it was not analyzed if the laureates are consuming more chocolate than average. And this connection would be drawn by the first approach. Therefore the experimental setup is very poorly chosen. The authors want to refute the null-hypothesis that there is no relationship between chocolate consume and intelligence by using data, which is logically not connected to the thesis.\\

Summarizing, the authors should not publish this insight without further investigations.\\

