
Question 1.3c:\\	
\textsl{A lab just finishes a randomized controlled trial on 10000 participants for a new drug, and find a treatment effect with p-value smaller than 0.05. After a journalist interviewed the lab, he wrote a news article titled "New trial shows strong effect of drug X on curing disease Y." Is this title appropriate? What about "New drug proves over 95\% success rate of drug X on curing disease Y"?}\\

Answer:\\
This is a dangerous example for the publication of scientific results. At first point it could be mentioned that the result was not a cure for a disease rather than a "treatment effect" of a new drug, which might be a difference. The second headline mistakes the p-value for the percentage of cases, in which the drug did not cure a disease. In both cases the journalist writes about study results which might be untrue.\\

Nevertheless, a too strict view on the p-value is sometimes the trigger of discussions. But in this case, a p-value of 0.05 is too high for studies relating to drug effectiveness. A better and in this context widely used choice should be a significance level of 0.01. The development and production is a very expensive procedure where decisions based on a weak foundation may lead to serious consequences. The safety of patients is another at least equally important point and reason to formulate stricter limits.\\


