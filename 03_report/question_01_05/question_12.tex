
Question 1.5.12:\\	
\textsl{In light of this paper, let's theoretically model the problem of concern in Problem 1.3! Suppose people base the decision to making scientific claim on p-values, which parameter does this influence? $R$, $\alpha$ or $\beta$? Describe the effect on the PPV if scientists probe random relations and just look at p-value as a certificate for making scientific conclusion.}\\

Answer:\\

The parameter $\alpha$ is chosen study-specific beforehand. $\beta$ describes the power or the chance of a type II error and is dependent from the data and $\alpha$. But the p-value influences the parameter $R$ because the type I error is accounted in it.\\

The PPV decreases as the number of random relations increases. If the p-value is the only criterion to judge whether a relationship is significant, a rate of about $\alpha$ is found (without corrections). Many of those are likely to be non-correctly evaluated. Therefore the PPV tends to be low.\\
