
Question 1.1.5:\\	
\textsl{In the randomized controlled trial, the children whose parents refused to participate in the trial got polio at the rate of 46 per 100000, while the children whose parents consented to participate got polio at a slighter higher rate of 49 per 100000 (treatment and control groups taken together). On the basis of these numbers, in the following year, some parents refused to allow their children to participate in the experiment and be exposed to this higher risk of polio. Were their conclusion correct? What would be the consequence if a large group of parents act this way in the next year's trial?}\\

Answer:\\
This conclusion is logically correct if only these two numbers would have been published. On the other hand the reduction of polio rate in the treatment group in comparison to the control group is considerable.\\

If a high stake of parents would behave like this in the next experiments, chances are that there are not enough participants to draw reliable conclusions from adequate sample sizes. This scenario would lead to poor results or even to the situation that the experiments cannot be executed. Thus the vaccination will not be tested enough to be released.\\

