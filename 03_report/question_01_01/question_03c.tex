
Question 1.1.3c:\\	
\textsl{Let us examine how reliable the estimates are for the NFIP study. A train of potentially problematic but quite possible scenarios cross your mind:\\Even if the act of “getting vaccine" does lead to reduced infection, it does not necessarily mean that it is the vaccine itself that leads to this result. Give an example of how this could be the case. Describe an aspect of experimental design that would eliminate biases not due to the vaccine itself. }\\

Answer:\\
A possible scenario of reducing the chance of getting infected with polio is being in a class full of vaccinated children. These are less likely to get infected and thus less likely to infect classmates. With an increasing rate of immune students the chance of getting an infection reduces.\\

This effect can be reduced by analyzing individuals infection history separated from other participants. For example only a small amount of students of individual schools could be asked to attend in the experiment. By considering a large amount of schools the necessary number of participants can be achieved.\\