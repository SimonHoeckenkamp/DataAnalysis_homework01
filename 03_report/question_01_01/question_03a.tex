
Question 1.1.3a:\\	
\textsl{Let us examine how reliable the estimates are for the NFIP study. A train of potentially problematic but quite possible scenarios cross your mind:\\Scenario: What if Grade 1 and Grade 3 students are different from Grade 2 students in some ways? For example, what if children of different ages are susceptible to polio in different degrees?\\Can such a difference influence the result from the NFIP experiment? If so, give an example of how a difference between the groups can influence the result. Describe an experimental design that will prevent this difference between groups from making the estimate not reliable.}\\

Answer:\\
The statistics cannot eliminate these doubts. If the test result would show separate numbers of the "control group" of grade 1 and 3 students, this effect could be indicated, but not be proven. Different factors could impair the immune system of the individuals in a different manner. Individual factors like age-differing immune answers could be a cause different polio rates. Other factors could be based on the environmental circumstances. For example there is a lesson in the second grade, which grade 1 or 3 students do not attend. Swimming lessons or demanding sports could have different effects on the health of students.\\

The best way to overcome these issues is the use of randomization when drawing the individuals for the different groups. Important is furthermore, that the age of the students, for which the experiment is made for, is precisely named in the publication.\\