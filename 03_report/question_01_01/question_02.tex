
Question 1.1.2:\\	
\textsl{For each of the NFIP study, and the Randomized controlled double blind experiment above, which numbers (or estimates) show the effectiveness of the vaccine? Describe whether the estimates suggest the vaccine is effective.}\\

Answer:\\
In case of the first study from 1954 the groups cannot be compared without doubts. On the one hand, the vaccinated group of grade 2 students seem to be comparable with the "control group" of grade 1 and 3 students, who did not get any vaccination. But the differing age of participants might have effects on the polio rate. This could be minimized, if the groups would consist of more comparable individuals. On the other hand participants could behave in a different manner if they knew about their vaccination. This could be reduced by a double-blind experiment.\\

There are more reliable conclusions possible from the second experiment. The treatment group can be compared with the control group, because the randomization and the double-blind experiment, minimizes the two effects described before.\\

Nevertheless, the polio rates of participants, who did not consent to the vaccination, and the vaccinated students seem similar in both experiments. But here again, the age structure is not considered (or mentioned), which decreases the reliability of conclusions.\\