
Question 1.1.3b:\\	
\textsl{Let us examine how reliable the estimates are for the NFIP study. A train of potentially problematic but quite possible scenarios cross your mind:\\Polio is an infectious disease. The NFIP study was not done blind; that is, the children know whether they get the vaccine or not. Could this bias the results? If so, Give an example of how it could bias the results. Describe an aspect of an experimental design that prevent this kind of bias.}\\

Answer:\\
The double-blindness is an important factor in experiments. Students are likely to behave in a different manner if they know that they are vaccinated, or not vaccinated respectively. They might be more cautious if they know, that they did not get a vaccination, or analog inversely less cautious if they did. This behavior could effect for example the number of friends students visit per week.\\

The students behavior can only be as comparable as possible if both, the treatment and the control group, think they are vaccinated. This is achieved by using a placebo next the corresponding drug. The experimenters as well should not know who gets a placebo or the drug to minimize the bias from this side of the experiment.\\